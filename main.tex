\documentclass[a4paper, 11pt, ngerman]{article}

\usepackage{geometry}
\usepackage{lipsum}
\usepackage{graphicx}
\usepackage{wrapfig}
%\usepackage[pdftex]{hyperref}
\usepackage[utf8]{inputenc}
\usepackage[T1]{fontenc}
\usepackage{lmodern}
\usepackage[ngerman]{babel}
\usepackage[figurewithin=section, 
		   font=small, 
		   labelfont=bf]
		   {caption}

\usepackage[babel,german=quotes]{csquotes}
\usepackage[backend=bibtex,style=authoryear]{biblatex}

\bibliography{quellen.bib}

\geometry{a4paper,
		top=25mm, 
		left=40mm, 
		right=25mm, 
		bottom=30mm, 
		headsep=10mm, 
		footskip=12mm}

\graphicspath{ {./images/} }

\pagestyle{plain}
\pagenumbering{arabic}

\addto\captionsngerman{
\renewcommand{\tablename}{\small{\textbf{Tab.}}}%
\renewcommand{\figurename}{\small{\textbf{Abb.}}}%
}

\newcommand*{\captionsource}[2]{%
  \caption[{#1}]{%
    #1%
    \\\hspace{\linewidth}%
    \textbf{Quelle:} #2%
  }%
}

\begin{titlepage}
	\title{\textbf{Camera Obscura} \\ Eine kurze Geschichte der Fotografie}
	\author{Markus Mayer}
	\date{\today}
\end{titlepage}

\begin{document}

\maketitle

\begin{abstract}
	Trololo
\end{abstract}

\tableofcontents
\clearpage

\section{Die Erfindung der Fotografie}

\subsection{Die Camera Obscura}

\begin{wrapfigure}{r}{0.5\textwidth}
  \vspace{-20pt}
  \begin{center}
    \includegraphics[width=0.48\textwidth]{camera_obscura-17-jh.jpg}
  \end{center}
  \vspace{-20pt}
  \captionsource{Camera Obscura, 17. Jh., außerdem ein richtig langer Text}{\cite{CamObs:2015:Online}}
  \vspace{-10pt}
\end{wrapfigure}

Sowas \cite{KleiGeFo:Book} nämlich.

\lipsum



\subsection{Heliographie}

Niépce

\lipsum

\subsection{Daguerreotypie}

Daguerre

\lipsum

\subsection{Carte de Visite}

\lipsum

\subsection{Bewegungsstudien}

Muybridges

\lipsum

\subsection{Das bewegte Bild}

Cinematographie

\lipsum

\subsection{Farbfotografie}

\lipsum

\section{Kunst und Alltag - die neue Fotografie}

\subsection{Die Fotografie als Kunstform}

Stieglitz

\lipsum

\subsection{The Kodak moment: Just press the button}

\lipsum

\subsection{Instant-Film}

Land Camera

\lipsum

\subsection{Vom Kleinbild zur Street-Fotografie}

\lipsum

\subsection{Fashion und Werbung}

\lipsum

\subsection{Kriegsfotografie}

Fotografie im Krieg

\lipsum

\subsection{Die digitale Fotografie}

\lipsum

\subsection{Selfie, Foodporn, Instragram}

Erstes Selfie anno 1750

Das (Selbst-)Portrait schon immer ein integraler Bestandteil der Kunst.

Katzen

\lipsum

\addcontentsline{toc}{section}{Abbildungsverzeichnis}
\listoffigures

\printbibliography

\end{document}