\documentclass[a4paper, 11pt, ngerman]{article}

\usepackage{eurosym}
\usepackage{geometry}
\usepackage{lipsum}
\usepackage{graphicx}
\usepackage{wrapfig}
%\usepackage[pdftex]{hyperref}
\usepackage[utf8]{inputenc}
\usepackage[T1]{fontenc}
\usepackage{lmodern}
\usepackage[ngerman]{babel}
\usepackage[figurewithin=section, 
		   font=small, 
		   labelfont=bf]
		   {caption}

\usepackage{varioref}
\usepackage{cleveref}
\usepackage{siunitx}
\usepackage{todonotes}
\usepackage{subcaption}

\usepackage[babel,german=quotes]{csquotes}
\usepackage[backend=bibtex,style=authoryear]{biblatex}

\bibliography{quellen.bib}

\geometry{a4paper,
		top=25mm, 
		left=40mm, 
		right=25mm, 
		bottom=30mm, 
		headsep=10mm, 
		footskip=12mm}

\graphicspath{ {./images/} }

\pagestyle{plain}
\pagenumbering{arabic}

\addto\captionsngerman{
\renewcommand{\tablename}{\small{\textbf{Tab.}}}%
\renewcommand{\figurename}{\small{\textbf{Abb.}}}%
}

\newcommand*{\captionsource}[2]{%
  \caption[{#1}]{%
    #1%
    \\\hspace{\linewidth}%
    \textbf{Quelle:} #2%
  }%
}

\newcommand{\name}[1]{\textsc{#1}}

\begin{titlepage}
	\title{\textbf{Camera Obscura} \\ Eine kurze Geschichte der Fotografie}
	\author{Markus Mayer}
	\date{\today}
\end{titlepage}

\begin{document}

\maketitle

\begin{abstract}
Die \textit{Camera Obscura}, jene \textit{dunkle Kammer}, die den Grundtypus eines jeden Apparates zur Erzeugung von Fotografien und bewegten Bildern beschreibt, ist mehr als nur der dunkle Raum, der das Licht besonders macht. Schon lange haben wir verlernt, den Prozess der Erzeugung eines Bildes, das Wesen der Fotografie als das komplexe Zusammenspiel aus Moment, Intuition, Technik und Substanz zu begreifen, das es ist. Zur sprichwörtlichen \textit{black box} ist die Fotografie für uns geworden, ein undurchsichtiger, oft sogar unsichtbarer Akt der Kopie unserer Realität, der uns zum Gott über die Maschine werden lässt und uns erlaubt, ohne Arbeit und ohne Zeit zu erschaffen -- zu kopieren, was die Natur uns zeigt. So trivial die Fotografie für uns geworden ist, so unachtsam wurden wir. Nicht selten erscheint sie uns realer als die Wirklichkeit -- und führt uns so hinters Licht.

Um die Bedeutung der Fotografie zu verstehen müssen wir ihre Geschichte kennen; diese Abhandlung soll sie dem Leser näher bringen.
\end{abstract}

\tableofcontents
\clearpage

\section{Der Ursprung der Fotografie}

Anders als die Malerei, in welcher der Künstler das Bild manuell durch Auftragen von Farbe auf ein Medium erzeugt, oder den Anfängen der Tiefdruckverfahren wie dem Kupferstich, in welchen das Bild in ein Medium geritzt wird, ist die Fotografie von jeher ein teilautomatisierter Prozess, bei dem der eigentliche Akt der Übertragung des Bildes nicht dem Künstler obliegt. Der Ursprung des Gerätes, der \textit{Kamera}, die diese Übertragung vornimmt, soll in \cref{sec:cameraobscure} beschrieben werden.

Wenn man ferner die Fotografie als Konsequenz der Malerei zur Kopie der Realität und Verewigung des Vergänglichen betrachtet, muss man das Augenmerk ebenfalls auf ihre zwei wichtigsten Vorfahren werfen. Die Herkunft der Landschaftsfotografie soll in \cref{sec:ursprung_landschaft} beschrieben werden, die der Portraitfotografie in \cref{sec:ursprung_portrait}.

\subsection{Die Camera Obscura}
\label{sec:cameraobscure}

Obschon die Entwicklung der Fotografie, wie wir sie heute kennen, erst in den letzten 200 Jahren stattfand, ist die Funktionsweise der \textit{Camera Obscura}, des dunklen Raumes zum Einfangen des Lichtes, schon länger bekannt.

\begin{wrapfigure}{r}{0.5\textwidth}
  \vspace{-20pt}
  \begin{center}
    \includegraphics[width=0.48\textwidth]{gemma-frisius_camera_obscura-1544}
  \end{center}
  \vspace{-20pt}
  \captionsource{Camera Obscura; Reiner Gemma-Frisius, 1544}{\cite{KleiGeFo:Book}}
  \label{fig:camob1544}
  \vspace{-50pt}
\end{wrapfigure}

Bereits in der \textit{Problemata physica} beschreibt \name{Aristoteles} (384-322 v. Chr.) das Phänomen des auf dem Kopf stehenden Bildes, wie es das Licht erzeugt, wenn es durch ein kleines Loch fällt -- einem Effekt, wie man ihn heutzutage etwa an Schlüssellöchern beobachten kann. \todo{Quelle}
 
\todo[inline]{http://www.fotoclub-ort.at/web/index.php/geschichte-der-fotografie/camera-obscura} 
 
Der italienische Maler und Ingenieur \name{Da Vinci} (1452-1519) beschäftigte sich mit dem Strahlengang des Lichtes und erkannte hierbei die Ähnlichkeit zur Funktionsweise des menschlichen Auges.

\begin{wrapfigure}{r}{0.5\textwidth}
  \vspace{-20pt}
  \begin{center}
    \includegraphics[width=0.48\textwidth]{Camera_obscura_Johann_Zahn}
  \end{center}
  \vspace{-20pt}
  \captionsource{"`Helle"' Camera Obscura nach Prinzip von Johann Zahn, 1685}{\cite{GesDerPho:Book}}
  \label{fig:camluc1685}
  \vspace{-10pt}
\end{wrapfigure}

Durch ihre indirekte Darstellung der Welt bot sich die Camera Obscura als ideales Mittel zur Beobachtung von Sonnenflecken und -finsternissen an und wurde für diesen Zweck bereits in der späten Antike und im frühen Mittelalter verwendet. Der niederländische Physiker \name{Reiner Gemma-Frisius} etwa beobachtete die Sonnenfinsternis 1544 mittels einer Camera Obscura nach \name{Aristoteles}'schem Prinzip und hielt dieses Ereignis in einer Zeichnung (\cref{fig:camob1544}) fest.

Im Zuge der aufkommenden Technologie des Linsenschleifens \todo{wann?} beschrieb der italienische \todo{was} \name{Daniele Barbaro} in seiner Schrift \textit{La pratica della prospeltiva} (sinng. \textit{Die Praxis der Perspektive}) \todo{quelle} eine verbesserte Kamera. Durch das Ersetzen des einfachen Loches durch eine Linse konnte nun die Qualität und Helligkeit der Abbildung drastisch verbessert werden. \todo{Quelle + Kepler scheint das zu kennen}

Nicht viel später, im Jahre 1685, \todo{quelle} konstruierte der \todo{was?} \name{Johann Zahn} die erste tragbare \textit{Camera Lucida} (helle Kammer) nach dem Prinzip der Camera Obscura: Hierbei wurde das Licht in der Kammer durch einen schräg angebrachten Spiegel auf eine Mattscheibe projiziert (\cref{fig:camluc1685}). Die kompakte Bauform und die Möglichkeit der Aufsicht auf das projizierte Bild erlaubte es nun, die Natur schlichtweg "`abzupausen"'. \todo{Bild} Ein solches Gerät soll etwa dem holländischen Maler \name{Jan Vermeer} als Hilfsmittel zur Erstellung seiner Gemälde gedient haben. \todo{Ansicht von Delft}

\subsection{Ursprung der Landschaftsfotografie}
\label{sec:ursprung_landschaft}

\begin{wrapfigure}{r}{0.5\textwidth}
  \vspace{-20pt}
  \begin{center}
    \includegraphics[width=0.48\textwidth]{Claude_Lorrain_-_Die_Verstossung_der_Hagar_(1668)}
  \end{center}
  \vspace{-20pt}
  \captionsource{Die Verstoßung der Hagar, Claude Lorrain, 1668}{\cite{VerstossungHagar:2015:Online}}
  \label{fig:verstossung1668}
  \vspace{-20pt}
\end{wrapfigure}

In der frühen mittelalterlichen Kunst Europas finden sich Darstellungen der Natur allein in Form von Bildnissen des Kloster- oder Hofgartens, als Nutz- oder Ziergarten. \todo{Quelle} Eine Abbildung der Landschaft außerhalb der Mauern des Klosters oder der Burg schien außer Frage. Obschon die Landschaft in der Malerei der Renaissance bildwichtiger Bestandteil zu werden beginnt, lässt sich noch in \name{Claude Lorrain}s \textit{Die Verstoßung der Hagar} (1668, \cref{fig:verstossung1668}) der Konflikt zwischen der Sicherheit des Innen -- so erdrückend es dort auch erscheint -- und der Ungewissheit des Außen erkennen. 

\begin{wrapfigure}{r}{0.5\textwidth}
  \vspace{-20pt}
  \begin{center}
    \includegraphics[width=0.48\textwidth]{Caspar_David_Friedrich_-_Der_Wanderer_ueber_dem_Nebelmeer}
  \end{center}
  \vspace{-20pt}
  \captionsource{Der Wanderer über dem Nebelmeer, Caspar David Friedrich, 1818}{\cite{WandererNebelmeer:2015:Online}}
  \label{fig:wanderer1818}
  \vspace{-5pt}
\end{wrapfigure}

Mit dem goldenen Zeitalter der Niederlande (ca. 1600-1700) beginnt ein exponentieller Zuwachs an Malereien, welche sich von biblischen Themen und den Motiven der Schäferdichtung abwenden, um die Landschaft selbst in den Vordergrund zu rücken. \todo{Quelle} Es ist letztlich die Zeit der Romantik (ca. 1800-1850), in der sich Maler wie \name{Caspar David Friedrich} finden, die mit Werken wie \textit{Der Wanderer über dem Nebelmeer} (1818, \cref{fig:wanderer1818}) die Schroffheit der Natur zum Kunstmotiv, zum Sehenswerten erheben (\cite{LandscapePainting:2015:Online}). \todo{Bessere Quelle finden}

Diese Bewegung ist es, welche die Vorlage für die imposanten Naturfoto- und -filmografien der Neuzeit bildet, wie man sie im Genre des Bergfilmes, etwa in \name{Arnold Fanck}s Szenerien von \textit{Der heilige Berg} und \textit{Die weiße Hölle vom Piz Palü} findet (vgl.~\cref{fig:heiligerBerg1926,fig:pizPalue1929}). 
Dieses neu gefundene Interesse für die Landschaft (und insbesondere die notwendige Schaffung der Infrastruktur für den Bergfilm der 1920er Jahre) bedeuteten nicht zuletzt auch die wesentlichen Grundvoraussetzungen für die Entwicklung des Tourismus, der heute mit der Fotografie untrennbar verbunden scheint.

\subsection{Ursprung der Portraitfotografie}
\label{sec:ursprung_portrait}

Im 18. und 19. Jahrhundert war die Malerei eines Portraits -- wie auch heute -- ein zeitaufwändiger Akt. \todo{Wie lange hat Vermeer gebraucht?} Entsprechend hoch muss der Preis für ein solches Bild gewesen sein\todo{wie viel?}, so dass nur wohlhabende Angehörige des Adels überhaupt in den Genuss eines Portraits ihrer Person kommen konnten; wer sich auf einem Portrait verewigt fand, hatte Geld, war wichtig.

Im Frankreich des frühen 19. Jahrhunderts, bedingt durch die Veränderung des Selbstbildes des Bürgertums in Folge der französischen Revolution, entstand das zunehmende Bedürfnis, sich selbst auf eine Stufe mit denen stellen zu können, die etwas bedeuteten, Portraits besaßen. Die bürgerliche Portraitmalerei suchte den Stil der höfischen Malerei nachzuahmen fand ihr Ziel in den leichten Farben und kleinen Formen der Miniaturmalerei. Im Marseille der 1850er Jahre waren 4-5 Miniaturmaler zugange, von denen zwei sich eines besseren Rufes erfreuten -- und dennoch verdienten diese Maler gerade genug, um ihren Lebensunterhalt zu bestreiten. Nur wenige Jahre später besaß Marseille 40-50 Fotografen, die jeder  bei einem mittleren Preis von 15 frs. (ca. 38 \euro) pro Fotografie knapp 15\,000 bis 18\,000 frs. (ca. 38\,000 bis 45\,500~\euro) im Jahr verdienten\footnote{Geschätzter Wechselkurs von Franc (1850) in Euro (2006) mittels \cite{Valeur:2015:Online}.} (vgl. \cite{PhoUndGes:Book}).

Zur Zeit Ludwig XV. entstand eine alternative Art des Portraitierens durch Ausschneiden des Gesichtsumrisses einer Person aus Glanzpapier. Nachdem sich der französische Finanzminister Étienne de Silhouette, 1759 seines Amtes entbunden, auf sein Anwesen zurückzog, erntete diese Technik den spöttischen Namen \textit{Silhouette} -- wie überhaupt alles, das billig war oder schattenhaft wirkte. Die Silhouettenschneiderei war kostengünstig, ohne besonderes Können durchzuführen und ging schnell vonstatten und begann so -- trotz ihrer Einfachheit -- die Miniaturmalerei vom Markt zu verdrängen.
Die Kommerzialisierung der Silhouetten fand ihren Anfang 1786 mit der Erfindung des \textit{Physionotrace} durch \name{Gilles-Louis Chrétien}. Mit diesem Gerät zum einfachen Nachziehen der Konturen bei gleichzeitiger Verkleinerung durch Hebelarme wurde die Zeit zur Erstellung eines Silhouetten-Stiches auf wenige Minuten reduziert. Dort, wo das Schneiden einer Silhouette noch ein gewisses handwerkliches Können bei der Nacharbeit der Konturen erforderte, benötigte der Physionotrace nicht einmal mehr das, wodurch der kommerziellen Erfolg des Gerätes gesichert war. Viele frühere Miniaturisten und Kupferstecher wandten sich nun der neuen Technik zu, die sich im Volk großer Beliebtheit erfreute. (\cite{PhoUndGes:Book}).

Während in der Miniaturmalerei der Künstler noch Eingriff in die Gestaltung des Bildes hatte und Besonderheiten der Person hervorheben oder abschwächen konnte, erlaubte der Physionotrace dies nicht. Er repräsentierte die pure, geradezu seelenlose Mechanisierung des Portraits. 
Wenngleich sich kein technischer Zusammenhang zwischen Camera Obscura und dem Physionotrace findet, stellt er durch die Einfachheit seiner Bedienung und der Natur seiner Automatisierung sehr wohl den spirituellen Vorgänger der Portraitfotografie dar --- und dennoch bedeutete die Technologie der Fotografie auch die Rückkehr der Kunst in die Mechanisierung des Portraits.


\section{Die (Er-)Findung der Fotografie}

Wie faszinierend die "`luzide"' Camera Obscura auf den Menschen gewirkt haben muss, zeigt \name{Peter Webber}s Film \textit{Das Mädchen mit dem Perlenohrring} eindrucksvoll\footnote{Ein Ausschnitt zur Camera Obscura ist unter \url{https://youtu.be/P8Xlo1GilbE} zu finden.}. Dabei hatte das Gerät einen entscheidenen Nachteil: Mit dem Licht verschwand auch das Bild. 

Im Jahre 1816 war es bereits bekannt, dass sich auf Chlorsilberpapier das Sonnenlicht festhalten ließ: Ein auf das Papier gelegtes Blatt würde später als helle Silhouette erscheinen -- jedoch konnte der Prozess nicht angehalten werden, so dass das Bild, im Licht betrachtet, schnell verdunkelte.
Es sollte nach der Erfindung \name{Zahn}s noch gute anderthalb Jahrhunderte dauern, bis die erste permanente Fotografie im Jahre 1826 das Licht der Welt erblickte. 

\subsection{Heliographie}

\name{Joseph Nicéphore Niépce}, 1765 als Sohn reicher Eltern in Chalon-sur Saône im Burgunder Land geboren, vertiefte sein Interesse an der Chemie mit dem Import der Technik der Lithographie nach Frankreich im Jahre 1814 (\cite{PhoUndGes:Book}).
%
\begin{wrapfigure}{r}{0.5\textwidth}
  \vspace{-20pt}
  \begin{center}
    \includegraphics[width=0.48\textwidth]{heliograph}
  \end{center}
  \vspace{-20pt}
  \captionsource{Die erste Fotografie: Ausblick auf den Gutshof Le Gras, Niépce, um 1827}{\cite{PhoUndGes:Book}}
  \label{fig:legras1826}
  \vspace{-5pt}
\end{wrapfigure}
%
Da sich die Beschaffung der notwendigen Materialien als schwierig herausstellte, begann er, zu experimentieren, wobei er schlussendlich Stein durch Metalle und Stift durch Licht ersetzte. 
1822 beherrschte er die Technik der Fixierung des Bildes; so gelang es ihm bereits, eine permanente Kopie auf einer Glasplatte herzustellen\footnote{Obschon es sich hierbei in der Tat um das erste dauerhafte fotografische Bild handelte, wird es im Allgemeinen nicht als solches betrachtet, da es durch Kopiertechniken entstand.}. 
Erst im Jahre 1826 war er jedoch in der Lage, seine Technik auf eine präparierte Zinnplatte anzuwenden: Hierbei beschichtete er die Platte mit Asphalt, der unter Sonneneinwirkung aushärtete. 
Mittels einer Mixtur aus Lavendelöl und Petrolium gelang es ihm nun, den unter Einfall weniger Lichtes ungehärteten Asphalt herauszulösen und erhielt so das erste beständige Direktpositivbild (\cref{fig:legras1826}) der Welt: Die Fotografie war geboren. \name{Niépce}s Technik erforderte eine Belichtung des Bildes über knapp acht Stunden, was ihr den Namen \textit{Heliographie} -- Sonnenmalerei -- einbrachte. 

\Cref{fig:legras1826} zeigt deutlich die zwei verschiedenen Schlagschatten, welche die Sonne im Verlauf der Zeit auf das Motiv warf. Diese langen Belichtungszeiten, die schlechte Erkennbarkeit des Bildes\footnote{Die hier dargestellte Abbildung ist optisch aufgebessert und hat mit dem Original hinsichtlich der Deutlichkeit nur wenig gemein.} und nicht zuletzt die schwierige Handhabung des Materials jedoch bedingten, dass die Technik kein kommerzieller Erfolg wurde. \name{Niépce}, der über die Zeit sein gesamtes Vermögen in die Entwicklung in die Heliographie steckte, verstarb in Armut und ohne Anerkennung seiner Leistungen. Es war sein Sohn \name{Isidore Niépce}, der zusammen mit \name{Louis Daguerre} den Durchbruch der Fotografie bewirkte.

\subsection{Daguerreotypie}

Der Maler \name{Louis Daguerre}, 1787 in Cormeille-en-Paris geboren und berühmt geworden durch die Erfindung des \textit{Dioramas}, hatte sich noch zu \name{Niépce}' Lebzeiten mit diesem in Verbindung gesetzt. 
Zusammen mit dessem Sohn \name{Isidore Niépce}, dessen einzige Erbschaft die Erfindung der Heliographie selbst war, versuchte er nun, das Verfahren zu monetarisieren. Wo \name{Niépce} scheiterte, scheiterten auch sie zunächst. 
Es war erst eine Weiterentwicklung des Verfahrens hin zu Silberhalogenid-beschichteten Glasplatten -- zur \textit{Daguerreotypie} -- und einem Pariser Gesetzesentwurf um 1839 zum staatlichen Ankauf der Technik der Fotografie, der ihnen zu Erfolg verhalf: 
Dieser Entwurf sprach \name{Daguerre}, der von jeher darauf bestand, bei allen Veröffentlichungen an erster Stelle genannt zu werden, sowie seinem Mitarbeiter \name{Niepce} lebenslänglichen Rente von 60000 frs. (ca. 142\,000 \euro) und 40000 frs. (ca. 95\,000 \euro) zu.
Der Antrag wurde ohne Gegenstimmen angenommen und das Verfahren am 19. August 1839 in der Académie des Sciences der Öffentlichkeit übergeben. Der Andrang zur Präsentation war gewaltig (\cite{PhoUndGes:Book}).

\begin{wrapfigure}{r}{0.5\textwidth}
  \vspace{-20pt}
  \begin{center}
    \includegraphics[width=0.48\textwidth]{daguerreotypie}
  \end{center}
  \vspace{-20pt}
  \captionsource{Frühe Daguerreotype und erste Fotografie eines lebenden Menschen: Boulevard du Temple, Paris, 3. Arrondissement; Daguerre, ca. April-Mai 1838}{\cite{GesDerPho:Book}}
  \label{fig:daguerre1838}
  \vspace{20pt}
\end{wrapfigure}
%
\name{Daguerre}, dessen Technik noch nicht reproduzierfähig war, erkannte die Gelegenheit und gründete eine Fabrik, welche die notwendigen Kameras, Glasplatten und Chemikalien in Serie produzierte.
%
\begin{wrapfigure}{r}{0.5\textwidth}
  \vspace{-20pt}
  \begin{center}
    \includegraphics[width=0.48\textwidth]{boulevard_du_temple_by_daguerre_detail}
  \end{center}
  \vspace{-20pt}
  \captionsource{Detail: Boulevard du Temple, Paris, 3. Arrondissement; Daguerre, ca. April-Mai 1838}{\cite{GesDerPho:Book}}
  \label{fig:daguerre_detail1838}
  \vspace{-10pt}
\end{wrapfigure}
%
Er lizensierte den Aufbau seiner rund 50 kg schweren Kameras und ließ diese vom Pariser Optiker \name{Giroux} zu einem Preis zwischen 300 und 400 frs. (ca. 700 bis 950 \euro) verkaufen.
Ende 1839 konstruierte der Baron \name{Séguier} eine verbesserten Kamera, welche nur noch 14 kg wog. 
In den folgenden Jahren fiel der Preis für die Fotoplatten von knapp 4 frs. (ca. 9,50 \euro) 1839 auf 1 frs. (ca. 2,40 \euro) im Jahre 1840 und bereits 1846 belief sich der jährliche Verkauf auf etwa 2000 Apparate und 500\,000 Platten (\cite{PhoUndGes:Book}).

\Cref{fig:daguerre1838} gilt als die erste Photographie eines lebenden Menschen. Durch die hohen Belichtungszeiten konnten nur unbewegte Objekte fotografisch festgehalten werden: Auf dem Bürgersteig (vgl. \cref{fig:daguerre_detail1838}) lässt sich eine Person bei einem Schuhputzer erkennen, die lange genug still hielt, um nicht zu verschwinden (\cite{GeschDerFo:Book,GesDerPho:Book}). Zwei verschiedene Daguerreotypien des Boulevards, die nacheinander aufgenommen wurden, sind in \Cref{fig:boulevard_mitohne1838} zu finden.

Während \name{Daguerre}s ursprünglicher Kameraentwurf noch Belichtungszeiten von 15 Minuten im grellem Licht bedingte, waren es die Entwürfe von \name{Chevalier}, \name{Petzval} und \name{Voigtländer}, welche die Belichtungszeiten auf 13 Minuten im Schatten (1840), 2-3 Minuten im Licht (1841) und sogar 40 Sekunden (1842) reduzierten.
Der Portraitfotografie stand damit nichts mehr im Wege.

\subsection{Kollodium-Nassplattenverfahren}

Eine drastische Verbesserung in puncto Reproduzierbarkeit und Bildqualität erzielten \name{F. S. Archer} und  \name{Gustave Le Gray} um 1850 mit ihrer Erfindung des Kollodium-Nassplattenverfahrens.
Hierbei wird eine polierte Glasplatte unmittelbar vor der Verwendung mit einem Gel aus Kollodium und Silbernitrat bestrichen und nach der Belichtung sofort entwickelt und (chemisch) fixiert.
Durch die Belichtung auf einen transparenten Träger war es nunmehr möglich, eine einmal aufgenommene Fotografie theoretisch beliebig oft zu kopieren.
Trotz des hohen Aufwandes des Verfahrens -- die (durchaus schweren) Glasplatten mussten zum Zielort transportiert, vor Ort vorbereitet und entwickelt werden -- erfreute es sich großer Beliebtheit und wurde aufgrund seiner Kostengünstigkeit und der Fähigkeit, mikroskopisch kleine Details abzubilden\footnote{Die Auflösung ist lediglich beschränkt durch die Größe eines Silberhalogenidkorns, aber abhängig vom Objektiv.} bis in die späten 1960er Jahre etwa im Bereich des Drucks industriell verwendet (\cite{CollodionProcess:2015:Online}).

\todo[inline]{Bild mit Pferdekutsche}

\subsection{Carte de Visite}

Einen weiteren Schritt zur Serienproduktion tätigte \name{André A. E.Disdéri} (1819-1889). Dieser erfand mit der mehrlinsigen Kamera, die es ermöglichte, binnen kurzer Zeit eine Serie von acht Bildern auf eine Kollodium-Nassplatte aufzunehmen, im Wesentlichen den Vorläufer der heutigen Passbildstreifen.
In einer kurzen Sitzung im Studio war es nun möglich, zwei Motive zu jeweils vier Variationen aufzunehmen und daraus die besten Bilder zu wählen.
Diese Technik, die er als \textit{Carte de Visite} -- Visitenkarte -- verkaufte, fand großen Anklang in der Bevölkerung. Hier ist es nun, dass sich die Miniaturmalerei und die Geschwindigkeit des Physionotrace zu einem erschwinglichen Preis in der Fotografie vereinen (vgl. \cref{sec:ursprung_portrait}).

\todo[inline]{Bild mit lachenden Menschen}

\subsection{Die Marke "`Nadar"' und die Fotografie als Kunst}

\name{Gaspard-Félix Tournachon}, 1820 in Paris geboren und zu Lebzeiten Karikaturist, Journalist, Schreiber, Fotograf und Aeronaut, schlug sich nach einem abgebrochenem Studium der Medizin im Alter von 18 Jahren in Lyon als Journalist durch. Hier schrieb er unter dem Künstlernamen \name{Nadar} kleinere Artikel für lokale Zeitschriften und knüfte über einen Verwandten Kontakte zu lokalen Künstlerkreisen und erlernte, mangels verfügbaren Budgets, die Zeichnerei als Autodidakt. 
Mit 22 Jahren zog er nach Paris, wo er für Zeitschriften wie die Vogue schrieb und auf den Geschmack der freien Künste kam. Mit 33 Jahren, durch Geldnot getrieben, entschloss er sich im Jahre 1853, ein photografisches Atelier in der Rue Saint-Lazare zu eröffnen.
Hier machten sich die früheren Kontaktaufnahmen und Arbeitsfelder bezahlt: Zu \name{Nadar} kam jeder, der in der Kunst, Literatur und Fotografie von Bedeutung war; \name{Nadar} traf die Künstlerelite von Paris (\cite{PhoUndGes:Book}).

Viele seiner Kunden waren Freunde, daher arbeitete \name{Nadar} zumeist ohne Entgelt. Dies spiegelt sich in seinen Werken wieder, die -- wie viele frühen Fotografieren -- einen authentischen, da nichtkommerziellen, künstlerischen Anspruch haben.
Seine Vorliebe für die Ballonfahrt entdeckte er um 1856: Hier gelang es ihm, die erste Luftaufnahme von Paris aus einem selbstgebauten Korbballon zu tätigen.
Auch ist es \name{Nadar}, dem die ersten erfolgreichen Experimente mit künstlichem Licht zuzuschreiben sind. So beginnt er um 1860, die Pariser Katakomben mithilfe von Magnesiumlampen fotografisch zu erschließen. "`[Die Katakomben sind] einer dieser Orte, die jeder sehen und niemand wiedersehen möchte"', beschreibt \name{Nadar} seine Besuche in der Pariser Unterwelt (\cite{GettyNadar:2015:Online}).

Ganz im Sinne der Nutzung der Camera Obscura, wie sie \name{Vermeer} zugeschrieben wird, dürften \name{Nadar}s Arbeiten der erste verbürgte Fall sein, in dem eine Fotografie als Vorlage für ein Gemälde herangezogen wurde. So ist es seine Fotografie der \name{Musette} (\cref{fig:musette1855}), bürgerlich \name{Marie-Christine Roux}, einem Pariser Modell, welche die Vorlage für \name{Ingres}' Gemälde \textit{La Source} (1856, \cref{fig:lasource1856}) ist.

\todo[inline]{Erste Luftaufnahmen}
\todo[inline]{Erste Verwendung von künstlichem Licht}
\todo[inline]{Fotostudio in Paris}

\subsection{Bewegungsstudien}

Mit dem Aufkommen der Elektrizität und der Papier-Filmrollen wurde es möglich, durch die Automatisierung des Auslösers der Kamera ganze Fotoserien zu erzeugen. \name{Eadweard Muybridge} (eigentlich \name{Edward James Muggeridge}, 1830-1904), beschäftigte sich um 1870 mit der Studie der Bewegung. Es gelingt ihm erstmals, mittels einer Reihe präzise ausgelöster Kameras, die Bewegung eines Objektes "`einzufrieren"'.
%
\begin{wrapfigure}{r}{0.5\textwidth}
  \vspace{-20pt}
  \begin{center}
    \includegraphics[width=0.48\textwidth]{animallocomotion}	
  \end{center}
  \vspace{-20pt}
%  \captionsource{Plate 626 in \textit{Animal Locomotion}; Muybridge, 1887}{\cite{GesDerPho:Book}}
	\caption{Plate 626 in \textit{Animal Locomotion}; Muybridge, 1887}
	\label{fig:annieg1887}
  \vspace{-10pt}
\end{wrapfigure}
\todo{Bildquelle}
%
So ist es \name{Muybridge}, der die Frage beantwortet, ob ein Pferd zu allen Zeiten den Boden berührt, oder ob es wohl fliegen kann. In seinem 1887 veröffentlichem Buch \textit{Animal Locomotion} ("`an electro-photographic
investigation of consecutive phases of animal movements"', welches jedoch auch Studien der menschlichen Bewegung beinhaltet), zeigt er als Platte 626 eine Bewegungsreihe der Vollblut-Stute Annie G., die diese Frage eindeutig beantwortet (\cref{fig:annieg1887}).

Spätestens hier beginnt die Fotografie nun Einzug in die Wissenschaften zu halten: Prozesse, für die das menschliche Auge zu langsam ist, sind für das Auge der Kamera problemlos zu erfassen, und wo der Mensch vergisst, kann die Fotografie konservieren. 
Nicht zuletzt ist es jedoch auch genau der serielle, automatische Charakter der Studien \name{Muybridge}s, welcher die Grundlage für das spätere bewegte Bild, den (Kino-)Film als solches, darstellt und es ist exakt die Technik der \textit{Animal Locomotion}, die den charakteristischen "`bullet time"'-Effekt moderner Filme wie \textit{Matrix} (1999) der \name{Wachowski}-Brüder oder Computerspielen wie \textit{Max Payne} (Remedy Entertainment 2001) prägt. 
Doch auch Einflüsse auf die Malerei den Zeichen- und Illustrationsstil lassen sich finden, wie etwa in \name{Duchamp}s Gemälde \textit{Akt, eine Treppe herabschreitend Nr.~2} (1912, vgl.~\cref{fig:duchamp1912}).

1874 rechtmäßig des Mordes an seiner Frau beschuldigt und wegen "`entschuldbaren Mordes"' freigesprochen entschließt sich \name{Muybridge} dann, unter dem Namen \name{Eduardo Santiago Muybridge} die Pazifikküste zu bereisen und widmet sich der Landschaftsfotografie und mechano-chemischen Experimenten\todo{Quelle}.

\clearpage
\subsection{Das bewegte Bild}

Spätestens nun ist der Weg zum Kinofilm geebnet. 
Apparate zur Darstellung bewegter Zeichnungen existieren bereits seit 1830, doch es ist eine Erfindung \name{Muybridge}s, die die Fotografie mit dem bewegten Bild vereint, ehe \name{Thomas Edison} mit dem \textit{Kinetographen} die Grundlage für das moderne Multiplex-Kino schafft.
%
\todo{Bildquelle}
\begin{wrapfigure}{r}{0.5\textwidth}
%  \vspace{-20pt}
  \begin{center}
    \includegraphics[width=0.48\textwidth]{zoopraxiscope1}	
  \end{center}
  \vspace{-20pt}
%  \captionsource{Zoopraxiskop; Muybridge, 1879}{\cite{GesDerPho:Book}}
	\caption{Zoopraxiskop; Muybridge, 1879}
	\label{fig:zoopraxiskop1879}
  \vspace{-10pt}
\end{wrapfigure}
%
Zwischen 1830 und 1833 betritt eine Erfindung \name{Joseph A. F. Plateau} (1801-1883) den Markt: Das \textit{Phenakistiskop}, der Augentäuscher, haucht Malereien Leben ein, lässt Bilder tanzen und erstaunt die Leute.
Eine mit Schlitzen versehene, bemalte Scheibe (vgl.~\cref{fig:phenakistiskop1893}) wird in Drehung versetzt und erlaubt dem durch die Schlitze blickenden Betrachter, in einem Spiegel, Leben zu sehen, wo keines ist. \todo{Quelle}

Nur zwei Jahre nach seinen Bewegungsstudien erkennt \name{Muybridge}, dass sich seine Technik auf das Phenakistiskop anwenden lässt und beschreibt 1879 das \textit{Zoopraxiskop}. 
Anstelle von Zeichnungen finden nun auch kolorierte Fotografien ihren Platz, die statt paralleler Darstellung wie in seinen Büchern nun bewegt sind.
\name{Muybridge} platziert ein helles Licht im Gerät, um die Scheibe zu durchleuchten, lässt das Bild durch ein Objektiv fallen und schafft somit den ersten Filmprojektor (\cref{fig:zoopraxiskop1879}).

\name{Thomas Edison}, berühmt geworden durch seine Kohlefaden-Glühbirne (1879), erweiterte das Konzept des Zoopraxiskops 1888 zum \textit{Kinetoskop}, welches dann von seinem Angestellten \name{William K. L. Dickson} 1889 erstmalig umgesetzt wurde.
%
\todo{Bildquelle}
\begin{wrapfigure}{r}{0.5\textwidth}
%  \vspace{-20pt}
  \begin{center}
    \includegraphics[width=0.48\textwidth]{KinetoscopeParlorbis}	
  \end{center}
  \vspace{-20pt}
%  \captionsource{Bacigalupi's Kinetoscope and phonograph parlor, Market Street, San Francisco, ca. 1894}{\cite{GesDerPho:Book}}
	\caption{Bacigalupi's Kinetoscope and phonograph parlor, Market Street, San Francisco, ca. 1894}
	\label{fig:sanfrancisco1894}
  \vspace{-10pt}
\end{wrapfigure}
%
Anstatt auf kompakte Filmscheiben verwendet \name{Edison} bis zu 20 Meter lange Filmrollen, deren Enden zu einem endlosen Kreislauf verklebt sind.
Diese bestehen aus knapp 35mm breiten Papierstreifen und stellen den Vorläufer des heute (2015) bereits aussterben Zelluloidfilms dar.

\todo[inline]{Schemazeichnung}

Es dauert nicht lange, \todo{wie lange?} bis eine erste Variante mit Tonspur erscheint, die den Ton in grafischer Form neben dem Bild auf die Filmrolle abbildet.
Der Andrang auf das neue Medium war so groß, dass nun ganze Kinetoskop-Salons gegründet wurden, wie der in \cref{fig:sanfrancisco1894} dargestellte Salon in San Francisco um ca. 1894, in welchem die Betrachtung von fünf Filmen den Gast 25 Cent kostete \todo{Quelle}.

\subsection{Farbfotografie}

Obschon die Geburt des echten Farbfilmes erst in die Zeit um 1935 fällt, als der \textit{Kodachrome} den Markt betritt, lassen sich die ersten Experimente mit Farbfotografien um die Jahre 1886 und 1905 einordnen.

\name{Gabriel Lippmann} (1845-1921), seines Zeichens luxemburgischer \todo{was denn?}, erfindet ein Verfahren zur fotografischen Farbreproduktion (vgl.~\cref{fig:lippmann1891}), für das er 1886 den Nobelpreis erhält\todo{wie funktioniert es?}. 

\todo{Bildquelle}
\begin{wrapfigure}{r}{0.25\textwidth}
  \vspace{-20pt}
  \begin{center}
    \includegraphics[width=0.22\textwidth]{Parrot_photo_made_by_Gabriel_Lippmann_in_1891}	
  \end{center}
  \vspace{-20pt}
%  \captionsource{Bacigalupi's Kinetoscope and phonograph parlor, Market Street, San Francisco, ca. 1894}{\cite{GesDerPho:Book}}
	\caption{Farbreproduktion eines Papageies von Gabriel Lippmann, 1891}
	\label{fig:lippmann1891}
  \vspace{-10pt}
\end{wrapfigure}
%
Weniger bedeutend, aber ungleich eindrucksvoller sind hingegen die Arbeiten \name{Sergey Prokudin-Gorsky}s. 1852 \todo{wo?} geboren, wird der russische Chemiker und Fotograf vom Zaren Nikolas II. beauftragt, das russische Reich fotografisch zu dokumentieren.
\name{Gorsky} strebt an, mittels seiner Technik zur Farbfotografie die Bildung russischer Schüler über die optische Farbprojektion zu erhöhen.

Da ihm nur die üblichen Mittel zur Schwarz-Weiß-Fotografie auf Kollodium-Nassplatten zur Verfügung steht, behilft er sich eines Tricks. Mittels einer eigens gebauten Kamerakonstruktion belichtet er dasselbe Motiv dreimal, wobei er für jede der Belichtungen einen anderen Farbfilter in den drei Primärfarben verwendet, den er in den Strahlengang einführt (vgl.~\cref{fig:gorsky1911}).
Die so aufgenommenen Bilder ließen sich nun, mit den entsprechenden Farbfiltern gleichzeitig auf dieselbe Fläche projiziert, zu einem Farbbild zusammenfügen.
Diese Arbeit ist es, die einen bemerkenswerten Einblick in das Russland des frühen 20. Jahrhunderts ermöglicht (vgl.~\cref{fig:gorsky1910}).
Das berühmteste Werk \name{Gorsky}s ist das Farbfoto \name{Leo Tolstoy}s, das ca. 1908 entstand (\cref{fig:gorsky1908}).

Als die Firma \name{Kodak} (siehe \cref{subsec:kodakmoment}) im Jahre 1935 schlussendlich den \textit{Kodachrome}-Film auf den Markt bringt, steht die Farbfotografie (siehe z.B.~\cref{fig:butterfield1949}) nun jedem offen, der eine Kleinbildkamera besitzt.

\section{Kunst und Alltag - die neue Fotografie}

\todo[inline]{PRESSEFOTOGRAFIE}

\subsection{The Kodak moment: Just press the button}
\label{subsec:kodakmoment}

1892 gründete \name{George Eastman}, \todo{was war er?} aus seiner Eastman Dry Plate Company die neue Firma \textit{Eastman Kodak Company} in Rochester, New York.
Sein Ziel war es, durch Serienproduktion günstiger Kameras den Massenmarkt zu erreichen.
Die aus dieser Idee enstandene Boxkamera \textit{Kodak Nr. 1} (1888-1895) sollte möglichst einfach zu bedienen und dabei möglichst billig sein. 
Gemäß dem Motto "`You press the button -- we do the rest"' \todo{Quelle} wurde die Kamera mit einem Film für 100 Aufnahmen vorgeladen und nach vollständiger Belichtung direkt an Kodak zurückgesandt, welche die Entwicklung des Filmes für den Kunden vornahm.

Mittels einer aggressiven Werbestrategie bei gleichzeitiger Überschwemmung des Marktes mit dem Produkt, setzte sich die Kamera als erste Rollfilmkamera\footnote{Rollfilm der Vorgänger des Kassettenfilmes, bei dem das Zelluloid frei auf eine Spindel aufgerollt ist.} und Beginn der Amateurfotografie im den Geist der Gesellschaft fest. 
\cite{Knipser:Book} widerlegt dies als Lüge: weder wurde "`Kodaks Rollfilm"' von Kodak erfunden, noch kennzeichnet die Kodak Nr. 1 den Beginn der Amateurfotografie, noch war die Kamera besonders günstig\footnote{Tatsächlich wurdem Rollfilmgeräte bereits 1854 erfunden und patentiert (\name{Spencer} und \name{Melhuish}), \name{Eastman} hielt lediglich ein Patent auf eine Halterung (\cite{Camerapedia:2014:Online}).}.

Unbestritten ist allerdings die Wirkung auf die Verbreitung der Fotografie, und so produziert Kodak nach und nach weitere Modelle, so etwa die Brownie-Serie ab 1900, die ihren Höhepunkt in den 1950er Jahren findet\todo{Werbebild}\todo{Quelle}. Diese Modelle sind so weit verbreitet, dass sie sich bereits 1928 den Namen "`Volkskamera"' \todo{Quelle} einbringen und damit den Begriff des "`Volksgerätes"' lange Zeit vor dem Volksempfänger (1938, \cite{Volksempfaenger:2015:Online}) und dem Volkswagen (1946, \cite{Volkswagen:2015:Online}) prägen.

Im Jahr 1925 produziert die deutsche Leitz Camera AG die erste Kleinbildfilm-Kamere \textit{Leica I}: An die Stelle des großen Rollfilmes tritt nun ein schmalerer Filmstreifen, der in einer Kassette geliefert wird.
Mit diesem ist es nun möglich, den Film auch bei Tageslicht zu wechseln; dies war zuvor bei Rollfilm durch den freiliegenden Filmstreifen nicht möglich.
Da jedoch die Produktions- und Beschaffungskosten der Filme, blieben die gewünschten Käufe zunächst aus. 
Dennoch schlossen sich bald konkurrierende Firmen an die Entwicklung an. So entwickelte Contax im Jahr 1932 ein erstes Kleinbildfilm-Modell, 1934 zog Kodak mit der \textit{Retina 1} nach. \todo{Quelle}
Aufgrund ihrer kompakteren Bauform setzten sich die Kleinbildkameras im Laufe der Zeit immer weiter durch und waren bald bis zu ihrer Ablösung durch digitale Kameramodelle der status quo im Amateur- und Pressefotografiebereich\footnote{Mittel- und Großformatkameras mit Roll- und Planfilm wurden und werden weiterhin im Mode- und Architekturfotografiebereich eingesetzt, jedoch allmählich durch spezielle Digitalkameras ersetzt. Dies geht einher mit der stetig geringer werdenden Anzahl an Filmunternehmen; So stellte Kodak seine Filmproduktion etwa bereits im Jahre 2009 ein.}.

\todo[inline]{wann war der Durchbruch?}

\subsection{Die Fotografie als Kunstform}

Die Retusche einer Fotografie ist keine Besonderheit moderner Software wie \textit{Photoshop}, die es dem Anwender geradezu lachhaft leicht macht, ein Bild aufzubessern oder geradeheraus zu manipulieren.
Gegenteilig sogar sind es viele analoge Reproduktionstechniken wie das \textit{Abwedeln} (aufhellen) oder \textit{Nachbelichten} (verdunkeln) eines Bildbestandteiles, sowie die manuelle Retusche mit Pinsel und Farbe, welche die modernen Methoden inspirierten. \todo{Verweis auf Retuschekapitel}
Es müssen diese Möglichkeiten in Kombination mit der Auflösung der Fotografie vom Elitären in die Allgemeinheit hinein gewesen sein, die den New Yorker \name{Alfred Stieglitz} (1864-1946) dazu bewegt haben, einen extremen Standpunkt einzunehmen.

\name{Stieglitz}, Herausgeber der Zeitschrift \textit{Camera Work}, setzte sich zum Ziel, die Fotografie als anerkannte Kunstform zu etablieren.
Hierbei entsagte er allen Methoden der Manipulation und forderte stattdessen, den Eingriff des Fotografen allein auf den Blick und die Wahl des Momentes zu beschränken.
Nur konsequent beschränkte Stieglitz seine Arbeit daher insbesondere auf die unwirtlichsten Momente, fotografierte bei Regen, Schnee und Nebel und suchte die Besonderheit im Moment selbst und ist dabei Malern wie \name{Caspar David Friedrich} (vgl.~\cref{sec:ursprung_landschaft}) in seiner Natur nicht ganz fremd.
\Cref{fig:stieglitz1903} zeigt eine Gegenüberstellung einer Fotografie \name{Stieglitz} im Jahr 1903 mit einer Fotografie \name{Steichen}s von 1904. 
Obschon beide dasselbe Motiv thematisieren, zeigt sich ein deutlicher Unterschied zwischen der Ästhetik des "`puren"' Bildes (\name{Stieglitz}) und der des retuschierten (\name{Steichen}).

Mit seinen Forderungen nimmt \name{Stieglitz} im Grunde vorweg, was die Street-Fotografie Jahre später prägen wird (\cref{sec:street}). Der pure, authentische, ungeschliffene Moment und ein ebenso ungeschliffenes Foto prägen die Ästhetik einer ganzen Generation von Bildern des späten 20. Jahrhunderts bis zu ihrer Ablösung durch den virtuellen Effektfilter der Handyfotografie mit dem Aufkommen des Smartphones um 2007.
Ironischerweise adaptiert die Amateurfotografie damit das, was \name{Stieglitz} zur Definition der Kunst erklärt und wirft die Frage nach der Kunst damit erneut auf.

\todo{Verweis auf Instagram}

\subsection{Land-Kameras und der Instant-Film}

Eine zweite Entwicklung und spiritueller Vorgänger der Digitalfotografie fand zwischen 1947 und 1983 ihren Höhepunkt.
\name{Edwin Herbert Land} reichte 1933 ein Patent für Polarisationsfolien ein, die er unter dem Namen \textit{Polaroid} u.a. als Filter in Sonnenbrillen vermarktete.
1947 stellte er dann der Optical Society of America ein neuartiges Kameramodell vor, welches mittels eines speziellen Trennbildfilmes in der Lage war, nur kurze Zeit nach der Belichtung ein fertig entwickeltes Bild zu produzieren.
Das erste Land-Kameramodell Typ 95 wurde schließlich im November 1948 in Boston verkauft. Zu den ursprünglich nur schwarz-weißen Filmen gesellten sich 1963 mit dem \textit{Polacolor} der erste Farb-Sofortbildfilm hinzu (\cite{Polaroid:2015:Online}).
Bilder aus Polaroid-Kameras waren bis zur Entwicklung neuerer Filmtypen Unikate; das Polaroid war, genau wie der Moment seiner Entstehung, einzigartig.

\todo[inline]{und das ist interessant, weil ...?}

\subsection{Vom Kleinbild zur Street-Fotografie}
\label{sec:street}

Geburt der Street Photography
- Henri Cartier-Bresson, 1908-2004
- Robert Frank, 1924-heute  „The Americans“, 1958
- Vivian Maier, 1926-2009
- Street ist
-- Authentisch, nah, direkt
-- Nicht immer ernst

\subsection{Fashion und Werbung}

\lipsum

\subsection{Kriegsfotografie}

Fotografie im Krieg

\lipsum

\subsection{Die digitale Fotografie}

\lipsum

\subsection{Selfie, Foodporn, Instragram}

Erstes Selfie anno 1750

Das (Selbst-)Portrait schon immer ein integraler Bestandteil der Kunst.

Katzen

\lipsum

\section{Zusätzliche Abbildungen}

\begin{figure}[h]
	\centering
	\begin{subfigure}[b]{0.5\textwidth}
	    \includegraphics[width=\textwidth]{der_heilige_berg}	
	    \caption{Der Heilige Berg (1926)}
	    \label{fig:heiligerBerg1926}
	\end{subfigure}%
	~
	\begin{subfigure}[b]{0.455\textwidth}
	    \includegraphics[width=\textwidth]{piz_palue_ausschnitt}	
	    \caption{Die weiße Hölle vom Piz Palü (1929)}
	    \label{fig:pizPalue1929}
	\end{subfigure}%

	\captionsource{Ausschnitte aus \textit{Der heilige Berg} und \textit{Die weiße Hölle vom Piz Palü}, Arnold Fanck}{\cite{MountainAndFog:2007:Online}}
\end{figure}

\begin{figure}[h]
	\centering
	\begin{subfigure}[b]{\textwidth}
	    \includegraphics[width=\textwidth]{daguerreotypie}	
	    \caption{Ansicht mit Mensch}
	    \label{fig:boulevard_mit1838}
	\end{subfigure}
	
	\begin{subfigure}[b]{\textwidth}
	    \includegraphics[width=\textwidth]{boulevard-without}	
	    \caption{Ansicht ohne Mensch}
	    \label{fig:boulevard_ohne1838}
	\end{subfigure}%

	\captionsource{Zwei Ansichten des Boulevard du Temple; Daguerre, ca. 1838}{\cite{GesDerPho:Book}}
	\label{fig:boulevard_mitohne1838}
\end{figure}


\begin{figure}[h]
	\centering
	\begin{subfigure}[b]{0.5\textwidth}
	    \includegraphics[width=\textwidth]{musette}	
	    \caption{Musette; Nadar 1855}
	    \label{fig:musette1855}
	\end{subfigure}%
	~
	\begin{subfigure}[b]{0.36\textwidth}
	    \includegraphics[width=\textwidth]{Jean_Auguste_Dominique_Ingres_-_The_Spring_-_Google_Art_Project_2}	
	    \caption{La Source; Ingres 1856}
	    \label{fig:lasource1856}
	\end{subfigure}%

	\captionsource{Die Fotografie \textit{Musette} als Vorlage für das Gemälde \textit{La Source}}{QUELLE FEHLT}
\end{figure}
\todo{Bildquelle}

\begin{figure}[h]
	\centering
	\begin{subfigure}[t]{0.445\textwidth}
	    \includegraphics[width=\textwidth]{30_07_Duchamp_01__Large_}	
	    \caption{Akt, eine Treppe herabschreitend Nr. 2; Duchamp 1912}
	\end{subfigure}%
	~
	\begin{subfigure}[t]{0.5\textwidth}
	    \includegraphics[width=\textwidth]{4422ca54a5}	
	    \caption{Duchamp selbst, eine Treppe herabsteigend}
	\end{subfigure}%

	\captionsource{Einfluss Muybridges auf die Kunst von Duchamp}{\cite{Duchamp:2011:Online}}
	\label{fig:duchamp1912}
\end{figure}

\begin{figure}[h]
	\centering
	\includegraphics[width=0.5\textwidth]{Eadweard_Muybridges_phenakistoscope,_1893}	
	\caption{Phenakistiskopscheibe; Muybridge, 1893}
	\label{fig:phenakistiskop1893}
\end{figure}
\todo{Bildquelle}

\begin{figure}[h]
	\centering
	\includegraphics[width=\textwidth]{1280px-Rgb-compose-Alim_Khan}	
	\caption{Seyyid Mir Mohammed Alim Khan; Sergey Prokudin-Gorsky, ca. 1911; Projiziertes Gesamtbild (links) und Einzelbilder (rechts).}
	\label{fig:gorsky1911}
\end{figure}
\todo{Bildquelle}

\begin{figure}[h]
	\centering
	\includegraphics[width=\textwidth]{Prokudin-Gorsky5}	
	\caption{Armenische Frau; Sergey Prokudin-Gorsky, ca. 1910}
	\label{fig:gorsky1910}
\end{figure}
\todo{Bildquelle}

\begin{figure}[h]
	\centering
	\includegraphics[width=0.75\textwidth]{LNTolstoy_Prokudin-Gorsky}	
	\caption{Leo Tolstoy; Sergey Prokudin-Gorsky, ca. 1908}
	\label{fig:gorsky1908}
\end{figure}
\todo{Bildquelle}

\begin{figure}[h]
	\centering
	\includegraphics[width=\textwidth]{London_,_Kodachrome_by_Chalmers_Butterfield_edit}	
	\caption{Farbfoto auf Kodachrome: Salesbury Avenue from Picadilly Circus; Chalmers Butterfield, ca. 1949}
	\label{fig:butterfield1949}
\end{figure}
\todo{Bildquelle}

\begin{figure}[h]
	\centering
	\begin{subfigure}[t]{0.42\textwidth}
	    \includegraphics[width=\textwidth]{flatiron_stieglitz}	
	    \caption{Flatiron Building; Stieglitz 1903}
	\end{subfigure}%
	~
	\begin{subfigure}[t]{0.5\textwidth}
	    \includegraphics[width=\textwidth]{Steichen_flatiron}	
	    \caption{Flatiron Building; Steichen 1904}
	\end{subfigure}%

	\captionsource{Gegenüberstellung der Arbeit Stieglitz und Steichen}{\cite{Duchamp:2011:Online}}
	\label{fig:stieglitz1903}
\end{figure}
\todo{Bildquelle}

\clearpage

\addcontentsline{toc}{section}{Abbildungsverzeichnis}
\listoffigures

\printbibliography

\end{document}