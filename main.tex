\documentclass[a4paper, 11pt, ngerman]{article}

\usepackage{eurosym}
\usepackage{geometry}
\usepackage{lipsum}
\usepackage{graphicx}
\usepackage{wrapfig}
%\usepackage[pdftex]{hyperref}
\usepackage[utf8]{inputenc}
\usepackage[T1]{fontenc}
\usepackage{lmodern}
\usepackage[ngerman]{babel}
\usepackage[figurewithin=section, 
		   font=small, 
		   labelfont=bf]
		   {caption}

\usepackage{varioref}
\usepackage{cleveref}
\usepackage{siunitx}
\usepackage{todonotes}
\usepackage{subcaption}

\usepackage[babel,german=quotes]{csquotes}
\usepackage[backend=bibtex,style=authoryear]{biblatex}

\bibliography{quellen.bib}

\geometry{a4paper,
		top=25mm, 
		left=40mm, 
		right=25mm, 
		bottom=30mm, 
		headsep=10mm, 
		footskip=12mm}

\graphicspath{ {./images/} }

\pagestyle{plain}
\pagenumbering{arabic}

\addto\captionsngerman{
\renewcommand{\tablename}{\small{\textbf{Tab.}}}%
\renewcommand{\figurename}{\small{\textbf{Abb.}}}%
}

\newcommand*{\captionsource}[2]{%
  \caption[{#1}]{%
    #1%
    \\\hspace{\linewidth}%
    \textbf{Quelle:} #2%
  }%
}

\newcommand{\name}[1]{\textsc{#1}}

\begin{titlepage}
	\title{\textbf{Camera Obscura} \\ Eine kurze Geschichte der Fotografie}
	\author{Markus Mayer}
	\date{\today}
\end{titlepage}

\begin{document}

\maketitle

\begin{abstract}
Die \textit{Camera Obscura}, jene \textit{dunkle Kammer}, die den Grundtypus eines jeden Apparates zur Erzeugung von Fotografien und bewegten Bildern beschreibt, ist mehr als nur der dunkle Raum, der das Licht besonders macht. Schon lange haben wir verlernt, den Prozess der Erzeugung eines Bildes, das Wesen der Fotografie als das komplexe Zusammenspiel aus Moment, Intuition, Technik und Substanz zu begreifen, das es ist. Zur sprichwörtlichen \textit{black box} ist die Fotografie für uns geworden, ein undurchsichtiger, oft sogar unsichtbarer Akt der Kopie unserer Realität, der uns zum Gott über die Maschine werden lässt und uns erlaubt, ohne Arbeit und ohne Zeit zu erschaffen -- zu kopieren, was die Natur uns zeigt. So trivial die Fotografie für uns geworden ist, so unachtsam wurden wir. Nicht selten erscheint sie uns realer als die Wirklichkeit -- und führt uns so hinters Licht.

Um die Bedeutung der Fotografie zu verstehen müssen wir ihre Geschichte kennen; diese Abhandlung soll sie dem Leser näher bringen.
\end{abstract}

\tableofcontents
\clearpage

\section{Der Ursprung der Fotografie}

Anders als die Malerei, in welcher der Künstler das Bild manuell durch Auftragen von Farbe auf ein Medium erzeugt, oder den Anfängen der Tiefdruckverfahren wie dem Kupferstich, in welchen das Bild in ein Medium geritzt wird, ist die Fotografie von jeher ein teilautomatisierter Prozess, bei dem der eigentliche Akt der Übertragung des Bildes nicht dem Künstler obliegt. Der Ursprung des Gerätes, der \textit{Kamera}, die diese Übertragung vornimmt, soll in \cref{sec:cameraobscure} beschrieben werden.

Wenn man ferner die Fotografie als Konsequenz der Malerei zur Kopie der Realität und Verewigung des Vergänglichen betrachtet, muss man das Augenmerk ebenfalls auf ihre zwei wichtigsten Vorfahren werfen. Die Herkunft der Landschaftsfotografie soll in \cref{sec:ursprung_landschaft} beschrieben werden, die der Portraitfotografie in \cref{sec:ursprung_portrait}.

\subsection{Die Camera Obscura}
\label{sec:cameraobscure}

Obschon die Entwicklung der Fotografie, wie wir sie heute kennen, erst in den letzten 200 Jahren stattfand, ist die Funktionsweise der \textit{Camera Obscura}, des dunklen Raumes zum Einfangen des Lichtes, schon länger bekannt.

\begin{wrapfigure}{r}{0.5\textwidth}
  \vspace{-20pt}
  \begin{center}
    \includegraphics[width=0.48\textwidth]{gemma-frisius_camera_obscura-1544}
  \end{center}
  \vspace{-20pt}
  \captionsource{Camera Obscura; Reiner Gemma-Frisius, 1544}{\cite{KleiGeFo:Book}}
  \label{fig:camob1544}
  \vspace{-50pt}
\end{wrapfigure}

Bereits in der \textit{Problemata physica} beschreibt \name{Aristoteles} (384-322 v. Chr.) das Phänomen des auf dem Kopf stehenden Bildes, wie es das Licht erzeugt, wenn es durch ein kleines Loch fällt -- einem Effekt, wie man ihn heutzutage etwa an Schlüssellöchern beobachten kann. \todo{Quelle}
 
\todo[inline]{http://www.fotoclub-ort.at/web/index.php/geschichte-der-fotografie/camera-obscura} 
 
Der italienische Maler und Ingenieur \name{Da Vinci} (1452-1519) beschäftigte sich mit dem Strahlengang des Lichtes und erkannte hierbei die Ähnlichkeit zur Funktionsweise des menschlichen Auges.

\begin{wrapfigure}{r}{0.5\textwidth}
  \vspace{-20pt}
  \begin{center}
    \includegraphics[width=0.48\textwidth]{Camera_obscura_Johann_Zahn}
  \end{center}
  \vspace{-20pt}
  \captionsource{"`Helle"' Camera Obscura nach Prinzip von Johann Zahn, 1685}{\cite{GesDerPho:Book}}
  \label{fig:camluc1685}
  \vspace{-10pt}
\end{wrapfigure}

Durch ihre indirekte Darstellung der Welt bot sich die Camera Obscura als ideales Mittel zur Beobachtung von Sonnenflecken und -finsternissen an und wurde für diesen Zweck bereits in der späten Antike und im frühen Mittelalter verwendet. Der niederländische Physiker \name{Reiner Gemma-Frisius} etwa beobachtete die Sonnenfinsternis 1544 mittels einer Camera Obscura nach \name{Aristoteles}'schem Prinzip und hielt dieses Ereignis in einer Zeichnung (\cref{fig:camob1544}) fest.

Im Zuge der aufkommenden Technologie des Linsenschleifens \todo{wann?} beschrieb der italienische \todo{was} \name{Daniele Barbaro} in seiner Schrift \textit{La pratica della prospeltiva} (sinng. \textit{Die Praxis der Perspektive}) \todo{quelle} eine verbesserte Kamera. Durch das Ersetzen des einfachen Loches durch eine Linse konnte nun die Qualität und Helligkeit der Abbildung drastisch verbessert werden. \todo{Quelle + Kepler scheint das zu kennen}

Nicht viel später, im Jahre 1685, \todo{quelle} konstruierte der \todo{was?} \name{Johann Zahn} die erste tragbare \textit{Camera Lucida} (helle Kammer) nach dem Prinzip der Camera Obscura: Hierbei wurde das Licht in der Kammer durch einen schräg angebrachten Spiegel auf eine Mattscheibe projiziert (\cref{fig:camluc1685}). Die kompakte Bauform und die Möglichkeit der Aufsicht auf das projizierte Bild erlaubte es nun, die Natur schlichtweg "`abzupausen"'. \todo{Bild} Ein solches Gerät soll etwa dem holländischen Maler \name{Jan Vermeer} als Hilfsmittel zur Erstellung seiner Gemälde gedient haben. \todo{Ansicht von Delft}

\subsection{Ursprung der Landschaftsfotografie}
\label{sec:ursprung_landschaft}

\begin{wrapfigure}{r}{0.5\textwidth}
  \vspace{-20pt}
  \begin{center}
    \includegraphics[width=0.48\textwidth]{Claude_Lorrain_-_Die_Verstossung_der_Hagar_(1668)}
  \end{center}
  \vspace{-20pt}
  \captionsource{Die Verstoßung der Hagar, Claude Lorrain, 1668}{\cite{VerstossungHagar:2015:Online}}
  \label{fig:verstossung1668}
  \vspace{-20pt}
\end{wrapfigure}

In der frühen mittelalterlichen Kunst Europas finden sich Darstellungen der Natur allein in Form von Bildnissen des Kloster- oder Hofgartens, als Nutz- oder Ziergarten. \todo{Quelle} Eine Abbildung der Landschaft außerhalb der Mauern des Klosters oder der Burg schien außer Frage. Obschon die Landschaft in der Malerei der Renaissance bildwichtiger Bestandteil zu werden beginnt, lässt sich noch in \name{Claude Lorrain}s \textit{Die Verstoßung der Hagar} (1668, \cref{fig:verstossung1668}) der Konflikt zwischen der Sicherheit des Innen -- so erdrückend es dort auch erscheint -- und der Ungewissheit des Außen erkennen. 

\begin{wrapfigure}{r}{0.5\textwidth}
  \vspace{-20pt}
  \begin{center}
    \includegraphics[width=0.48\textwidth]{Caspar_David_Friedrich_-_Der_Wanderer_ueber_dem_Nebelmeer}
  \end{center}
  \vspace{-20pt}
  \captionsource{Der Wanderer über dem Nebelmeer, Caspar David Friedrich, 1818}{\cite{WandererNebelmeer:2015:Online}}
  \label{fig:wanderer1818}
  \vspace{-5pt}
\end{wrapfigure}

Mit dem goldenen Zeitalter der Niederlande (ca. 1600-1700) beginnt ein exponentieller Zuwachs an Malereien, welche sich von biblischen Themen und den Motiven der Schäferdichtung abwenden, um die Landschaft selbst in den Vordergrund zu rücken. \todo{Quelle} Es ist letztlich die Zeit der Romantik (ca. 1800-1850), in der sich Maler wie \name{Caspar David Friedrich} finden, die mit Werken wie \textit{Der Wanderer über dem Nebelmeer} (1818, \cref{fig:wanderer1818}) die Schroffheit der Natur zum Kunstmotiv, zum Sehenswerten erheben (\cite{LandscapePainting:2015:Online}). \todo{Bessere Quelle finden}

Diese Bewegung ist es, welche die Vorlage für die imposanten Naturfoto- und -filmografien der Neuzeit bildet, wie man sie im Genre des Bergfilmes, etwa in \name{Arnold Fanck}s Szenerien von \textit{Der heilige Berg} und \textit{Die weiße Hölle vom Piz Palü}) findet (vgl.~\cref{fig:heiligerBerg1926,fig:pizPalue1929}). 
Dieses neu gefundene Interesse für die Landschaft (und insbesondere die notwendige Schaffung der Infrastruktur für den Bergfilm der 1920er Jahre) bedeuteten nicht zuletzt auch die wesentlichen Grundvoraussetzungen für die Entwicklung des Tourismus, der heute mit der Fotografie untrennbar verbunden scheint.

\subsection{Ursprung der Portraitfotografie}
\label{sec:ursprung_portrait}

Im 18. und 19. Jahrhundert war die Malerei eines Portraits -- wie auch heute -- ein zeitaufwändiger Akt. \todo{Wie lange hat Vermeer gebraucht?} Entsprechend hoch muss der Preis für ein solches Bild gewesen sein\todo{wie viel?}, so dass nur wohlhabende Angehörige des Adels überhaupt in den Genuss eines Portraits ihrer Person kommen konnten; wer sich auf einem Portrait verewigt fand, hatte Geld, war wichtig.

Im Frankreich des frühen 19. Jahrhunderts, bedingt durch die Veränderung des Selbstbildes des Bürgertums in Folge der französischen Revolution, entstand das zunehmende Bedürfnis, sich selbst auf eine Stufe mit denen stellen zu können, die etwas bedeuteten, Portraits besaßen. Die bürgerliche Portraitmalerei suchte den Stil der höfischen Malerei nachzuahmen fand ihr Ziel in den leichten Farben und kleinen Formen der Miniaturmalerei. Im Marseille der 1850er Jahre waren 4-5 Miniaturmaler zugange, von denen zwei sich eines besseren Rufes erfreuten -- und dennoch verdienten diese Maler gerade genug, um ihren Lebensunterhalt zu bestreiten. Nur wenige Jahre später besaß Marseille 40-50 Fotografen, die jeder  bei einem mittleren Preis von 15 frs. (ca. 38 \euro) pro Fotografie knapp 15\,000 bis 18\,000 frs. (ca. 38\,000 bis 45\,500~\euro) im Jahr verdienten\footnote{Geschätzter Wechselkurs von Franc (1850) in Euro (2006) mittels \cite{Valeur:2015:Online}.} (vgl. \cite{PhoUndGes:Book}).

Zur Zeit Ludwig XV. entstand eine alternative Art des Portraitierens durch Ausschneiden des Gesichtsumrisses einer Person aus Glanzpapier. Nachdem sich der französische Finanzminister Étienne de Silhouette, 1759 seines Amtes entbunden, auf sein Anwesen zurückzog, erntete diese Technik den spöttischen Namen \textit{Silhouette} -- wie überhaupt alles, das billig war oder schattenhaft wirkte. Die Silhouettenschneiderei war kostengünstig, ohne besonderes Können durchzuführen und ging schnell vonstatten und begann so -- trotz ihrer Einfachheit -- die Miniaturmalerei vom Markt zu verdrängen.
Die Kommerzialisierung der Silhouetten fand ihren Anfang 1786 mit der Erfindung des \textit{Physionotrace} durch \name{Gilles-Louis Chrétien}. Mit diesem Gerät zum einfachen Nachziehen der Konturen bei gleichzeitiger Verkleinerung durch Hebelarme wurde die Zeit zur Erstellung eines Silhouetten-Stiches auf wenige Minuten reduziert. Dort, wo das Schneiden einer Silhouette noch ein gewisses handwerkliches Können bei der Nacharbeit der Konturen erforderte, benötigte der Physionotrace nicht einmal mehr das, wodurch der kommerziellen Erfolg des Gerätes gesichert war. Viele frühere Miniaturisten und Kupferstecher wandten sich nun der neuen Technik zu, die sich im Volk großer Beliebtheit erfreute. (\cite{PhoUndGes:Book}).

Während in der Miniaturmalerei der Künstler noch Eingriff in die Gestaltung des Bildes hatte und Besonderheiten der Person hervorheben oder abschwächen konnte, erlaubte der Physionotrace dies nicht. Er repräsentierte die pure, geradezu seelenlose Mechanisierung des Portraits. 
Wenngleich sich kein technischer Zusammenhang zwischen Camera Obscura und dem Physionotrace findet, stellt er durch die Einfachheit seiner Bedienung und der Natur seiner Automatisierung sehr wohl den spirituellen Vorgänger der Portraitfotografie dar --- und dennoch bedeutete die Technologie der Fotografie auch die Rückkehr der Kunst in die Mechanisierung des Portraits.


\section{Die (Er-)Findung der Fotografie}

\subsection{Heliographie}

Joseph Nicéphore \name{Niépce}

Erstes permanentes Foto 1826 \textit{Geburtsstunde der Fotografie}

Direktpositiv

Belichtung auf $\SI{21}{\centi\meter} \times \SI{16}{\centi\meter}$ große Zinnplatte mit Asphaltbeschichtung, Härtung unter Sonnenlicht. Lavendelöl und Petrolium zum Lösen des ungehärteten Asphalts.

\subsection{Daguerreotypie}

Louis Daguerre

Silberhalogenidplatten

Erste kommerzielle Fotografie

Serienproduktion

Verkürzung der Belichtungszeit durch Verbesserung des Objektivaufbaus

Josef \name{Petzval}, ca 1840

\subsection{Kollodium-Nassplattenverfahren}

F. S. \name{Archer} \& Gustave \name{Le Gray}

Polierte Glasplatte + Kollodium + Silbernitrat

Nasse Platte wird sofort belichtet und anschließend sofort verarbeitet.

\subsection{Carte de Visite}

André A.E. \name{Disdéri} (1819-1889)

Mehrlinsige Kamera erzeugt Serie von acht Bildern auf Kollodium-Nassplatte

\subsection{Nadar}

Gaspard-Félix \name{Tournachon} (Künstlername \name{Nadar})

Fotograf, Karikaturist, Journalist, Schreiber und Ballonfahrer

Erste Luftaufnahmen

Erste Verwendung von künstlichem Licht

Fotostudio in Paris

Musette (Marie-Christine Roux), \name{Nadar} 1855
La Source, \name{Ingres} 1856

\subsection{Bewegungsstudien}

Eadweard \name{Muybridge} (1830-1904)

Landschaftsfotografien

Bewegungsstudien ca. 1870

Animal Locomotion ca. 1887 

\begin{quote} an electro-photographic
investigation of consecutive
phases of animal movements
\end{quote}


\subsection{Das bewegte Bild}

\subsubsection{Phenakistiskop}

„Augentäuscher“, ca. 1830 bis 1833
Joseph A. F. \name{Plateau} (1801-1883)

\subsubsection{Zoopraxiskop}

\name{Muybridge}, 1879

\subsubsection{Kinetoskop}

Konzept von Thomas \name{Edison}, 1888

William K. L. \name{Dickson}, 1889
– Edisons Angestellter

Erster Filmprojektor
– ca. ~20m Endlosfilm
– ca. ~35mm Breite


Variante mit Tonspur
– Kinetograph

Kinetoskop-Salon in San Francisco, ca. 1894

\subsection{Farbfotografie}

Erste Farbfotografie

Gabriel \name{Lippmann}, 1845-1921, Luxemburg

- Nobelpreis für Farbreproduktions-Methode, 1886

Sergey \name{Prokudin-Gorsky}

1852-1944, russ. Chemiker und Fotograf

Dokumentation des russ. Reiches 1905 zwecks Schulbildung über Optische Farbprojektion

Berühmtestes Werk: Farbfoto von Leo Tolstoy, ca. 1908

Kodachrome, 1935 \cref{subsec:kodakmoment}

\section{Kunst und Alltag - die neue Fotografie}

\todo[inline]{PRESSEFOTOGRAFIE}

\subsection{Die Fotografie als Kunstform}

Alfred \name{Stieglitz}, 1864-1946
Camera Work

Etablierung der Fotografie als
Kunstform

Ablehnung von Manipulationen (Retusche)

Fotografie bei Regen, Schnee, Nebel, …

Flatiron Building, \name{Stieglitz} 1903 
Flatiron Building, \name{Steichen} 1904

\subsection{The Kodak moment: Just press the button}
\label{subsec:kodakmoment}

\textit{Kodak Nr. 1} (1888-1895)

Kodak Nr. 1
– Erste Rollfilmkamera
– Beginn der

Amateurfotografie
– Billiger Einstieg

Haken:
– Alles gelogen (Timm Starl; Knipser: Exkurs: Die Kodak-Legende)


 Aber:
– Werbung funktioniert!

Leitz Camera AG, 1925

\subsection{Instant-Film}

Land Camera

\lipsum

\subsection{Vom Kleinbild zur Street-Fotografie}

Geburt der Street Photography
- Henri Cartier-Bresson, 1908-2004
- Robert Frank, 1924-heute  „The Americans“, 1958
- Vivian Maier, 1926-2009
- Street ist
-- Authentisch, nah, direkt
-- Nicht immer ernst

\subsection{Fashion und Werbung}

\lipsum

\subsection{Kriegsfotografie}

Fotografie im Krieg

\lipsum

\subsection{Die digitale Fotografie}

\lipsum

\subsection{Selfie, Foodporn, Instragram}

Erstes Selfie anno 1750

Das (Selbst-)Portrait schon immer ein integraler Bestandteil der Kunst.

Katzen

\lipsum

\section{Zusätzliche Abbildungen}

\begin{figure}[h]
	\centering
	\begin{subfigure}[b]{0.5\textwidth}
	    \includegraphics[width=\textwidth]{der_heilige_berg}	
	    \caption{Der Heilige Berg (1926)}
	    \label{fig:heiligerBerg1926}
	\end{subfigure}%
	~
	\begin{subfigure}[b]{0.5\textwidth}
	    \includegraphics[width=\textwidth]{piz_palue_ausschnitt}	
	    \caption{Die weiße Hölle vom Piz Palü (1929)}
	    \label{fig:pizPalue1929}
	\end{subfigure}%

	\captionsource{Ausschnitte aus \textit{Der heilige Berg} und \textit{Die weiße Hölle vom Piz Palü}, Arnold Fanck}{\cite{MountainAndFog:2007:Online}}
\end{figure}

\addcontentsline{toc}{section}{Abbildungsverzeichnis}
\listoffigures

\printbibliography

\end{document}